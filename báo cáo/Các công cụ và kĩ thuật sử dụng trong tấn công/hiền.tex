\subsection{Tấn công Chọn văn bản (Chosen-Plaintext Attack)}
\subsubsection{Chosen-Plaintext Attack là gì?}
Tấn công bằng văn bản gốc được chọn là phương pháp mà tin tặc sử dụng để phá mã bí mật hoặc hệ thống mã hóa nhằm có được quyền truy cập trái phép vào thông tin. Trong cuộc tấn công này, tin tặc có thể chọn các tin nhắn cụ thể (bản rõ) để mã hóa bằng thuật toán mã hóa của mục tiêu. Làm như vậy sẽ tạo ra các bản mã tương ứng (phiên bản được mã hóa của bản rõ). Tin tặc phân tích các bản mã để tiết lộ toàn bộ hoặc một phần khóa mã hóa bí mật. Cuộc tấn công này nhằm mục đích thu thập thông tin có thể được sử dụng để giải mã các tin nhắn trong tương lai và xâm phạm tính bảo mật của hệ thống \cite{nordvpn-2024}.
\subsubsection{Tiến trình công cuộc tấn công}
\begin{itemize}
    \item Kẻ tấn công chọn bản rõ mà chúng muốn mã hóa bằng thuật toán hoặc hệ thống mã hóa mục tiêu.\cite{nordvpn-2024}
    \item Các bản rõ đã chọn sẽ được mã hóa bằng thuật toán mã hóa đích, tạo ra các bản mã tương ứng. Kẻ tấn công có thể tìm ra thuật toán mã hóa mục tiêu thông qua kiến thức công cộng hoặc kỹ thuật đảo ngược \cite{nordvpn-2024}.
    \item Kẻ tấn công quan sát và ghi lại các bản mã được tạo ra cho các bản rõ đã chọn \cite{nordvpn-2024}.
    \item Phân tích mối quan hệ giữa các bản rõ đã chọn và các bản mã tương ứng để xác định các mẫu hoặc lỗ hổng trong thuật toán mã hóa \cite{nordvpn-2024}.
    \item Dựa trên phân tích, kẻ tấn công có thể cố gắng tìm hiểu thêm về khóa bí mật hoặc khai thác bất kỳ điểm yếu nào được phát hiện trong quá trình mã hóa \cite{nordvpn-2024}.
    \item Nếu cần, kẻ tấn công có thể lặp lại các bước trên với các bản rõ được chọn bổ sung để thu thập thêm thông tin \cite{nordvpn-2024}.
\end{itemize}
\subsubsection{Các kiểu tấn công CPA}
\begin{itemize}
    \item \textbf{Batch chosen plaintext attack}: Tấn công hàng loạt bản rõ được chọn cho phép kẻ tấn công xử lý nhiều bản rõ cùng một lúc \cite{nordvpn-2024}.
    \item \textbf{Adaptive chosen plaintext attack}. Trong cuộc tấn công này, kẻ tấn công có thể tự động điều chỉnh các bản rõ đã chọn dựa trên phản hồi nhận được từ hệ thống mã hóa \cite{nordvpn-2024}.
\end{itemize}
\subsubsection{Nơi có thể xảy ra tấn công CPA}
\begin{itemize}
    \item Các giao thức mã hóa. Các cuộc tấn công bằng văn bản gốc được chọn có thể nhắm mục tiêu vào các giao thức mã hóa (ví dụ: liên lạc an toàn, trao đổi khóa hoặc xác thực). Bằng cách thao tác các bản rõ đã chọn trong quá trình thực thi giao thức, kẻ tấn công có thể hiểu rõ hơn về các lỗ hổng bảo mật của giao thức hoặc truy cập thông tin nhạy cảm \cite{nordvpn-2024}.
    \item An ninh mạng. Các cuộc tấn công bằng văn bản gốc được chọn có thể nhắm mục tiêu vào các hệ thống mật mã bảo vệ thông tin liên lạc trên mạng, chẳng hạn như VPN (mạng riêng ảo) \cite{nordvpn-2024}.
\end{itemize}
\subsubsection{Chosen-plaintext Attacks và Known-plaintext Attacks}
Chosen-plaintext Attacks liên quan đến việc đối thủ chọn bản rõ và phân tích bản mã tương ứng, trong khi Known-plaintext Attacks xảy ra khi kẻ tấn công sở hữu một phần kiến thức về bản rõ \cite{hoang-2023}.\\
Hiểu được sự khác biệt giữa hai cuộc tấn công mật mã này là rất quan trọng đối với các chiến lược bảo vệ mật mã hiệu quả \cite{hoang-2023}.
\begin{table}[H]
\resizebox{\textwidth}{!}{%
\begin{tabular}{|l|l|l|}
\hline
                                             & \textbf{Chosen-plaintext Attacks}      & \textbf{Known-plaintext Attacks}       \\ \hline
\multicolumn{1}{|l|}{\textbf{Bối cảnh}} &
  \multicolumn{1}{l|}{Kẻ tấn công lựa chọn plaintext} &
  \multicolumn{1}{l|}{Kẻ tấn công chỉ biết một vài plaintext} \\ \hline
\multicolumn{1}{|l|}{\textbf{Loại Attack}}   & \multicolumn{1}{l|}{Nghe lén thụ động} & \multicolumn{1}{l|}{Thao tác chủ động} \\ \hline
\multicolumn{1}{|l|}{\textbf{Loại hình}} &
  \multicolumn{1}{l|}{Giải mã văn bản mật mã} &
  \multicolumn{1}{l|}{Xác định phương pháp mã hoá} \\ \hline
\multicolumn{1}{|l|}{\textbf{Trình độ hiểu biết cần có}} &
  \multicolumn{1}{l|}{Không yêu cầu có hiểu biết về plaintext} &
  \multicolumn{1}{l|}{Một số plaintext cần hiểu rõ} \\ \hline
\multicolumn{1}{|l|}{\textbf{Tính phức tạp}} & \multicolumn{1}{l|}{Vừa phải}          & \multicolumn{1}{l|}{Thấp}              \\ \hline
\textbf{Ví dụ}                               & Giải mã cổ điển                        & Phân tích tần số                       \\ \hline
\end{tabular}%
}
\end{table}
Phân tích tần số tập trung vào việc kiểm tra sự xuất hiện của các chữ cái hoặc ký hiệu để xác định thuật toán mã hóa, không giống như phân tích mật mã cổ điển, kiểm tra văn bản mã hóa để tìm các mẫu và sai sót.
\subsubsection{Biện pháp bảo vệ và ngăn chặn tấn công}
\begin{itemize}
    \item Để ngăn chặn các cuộc tấn công bằng văn bản gốc đã chọn, điều cần thiết là sử dụng các khóa mã hóa mạnh được tạo ngẫu nhiên, đủ dài, được lưu trữ an toàn và đảm bảo chỉ những người dùng được ủy quyền mới có thể truy cập chúng.
    \item Các khóa không được sử dụng lại hoặc chia sẻ trên các hệ thống hoặc giao thức mã hóa khác nhau. \item Chìa khóa cũng cần được thay thường xuyên và vứt bỏ sau khi sử dụng. 
    \item Hơn nữa, điều cần thiết là phải sử dụng các thuật toán và chế độ mã hóa an toàn để đảm bảo tính bảo mật và tính toàn vẹn. Các thuật toán mã hóa được kiểm duyệt kỹ lưỡng sẽ trải qua quá trình thử nghiệm rộng rãi để chống lại các cuộc tấn công đã biết, bao gồm cả Chosen Plaintext Attack, chẳng hạn như AES-GCM hoặc ChaCha20-Poly1305. Các thuật toán và chế độ này sử dụng thẻ xác thực và mã hóa không dựa trên cơ sở để ngăn chặn các cuộc tấn công bằng văn bản gốc đã chọn.
\end{itemize}
\subsection{Tấn công Timing Attack}
\subsubsection{Tấn công Timing Attack là gì?}
\begin{itemize}
    \item Là một dạng tấn công thuộc loại side-channel mà hacker dựa vào thông tin phân tích được từ thời gian thực thi một đoạn logic từ hệ thống từ đó truy ra dần kết quả sau cùng, có thể xem Timing Attack là một dạng \textit{manual brute force} cũng được, vì cùng một cách thức để truy ra kết quả, nhưng khác là hacker cần đầu tư nhiều hơn để phân tích dữ liệu đang có để cho ra input tiếp theo \cite{trinh-2023}.
    \item Hacker luôn cố gắng lợi dụng bản năng của người lập trình viên, luôn muốn code của mình chạy càng nhanh càng tốt, để thực hiện ý đồ tấn công. Kẻ tấn công có thể bắt đầu với phỏng đoán (giả thiết) bit đầu tiên có thể là 0 hoặc là 1. Sau đó xem xét các kết quả giả thiết trong mối tương quan chặt chẽ giữa thời gian thực thi thực tế và thời gian dự đoán. Quá trình này được thực hiện nhiều lần, cho đến khi kẻ tấn công tìm ra được hết các bit khóa bằng cách quan sát sự tương quan về mặt thời gian giữa nhiều mẫu và bit khóa lựa chọn. Như vậy không gian tìm kiếm khóa có thể đã được thu nhỏ lại tấn công này được cho là khá đơn giản về mặt tính toán \cite{trinh-2023}.
\end{itemize}
\underline{\textbf{Ví dụ thực tế:}}\\
Đối với lập trình viên, chúng ta sẽ phải gặp trường hợp cần xác thực request đến từ phía client, hay server to server. Có rất nhiều cách để triển khai, thông thường là sử dụng JWT, Oauth2, OpenID Connect hay sử dụng bên thứ 3 để authen như Google, Facebook, AWS Cognito... Một trong những cách phổ biến và dễ triển khai nhất là sử dụng API key.\\
Tuy nhiên APIKey vẫn có thể bị lộ bằng nhiều cách tấn công, một trong số đó là Timing attack \cite{leo-2024}.
\subsubsection{Cách thức hoạt động}
Các cuộc tấn công tính thời gian lợi dụng thực tế là các thông tin đầu vào khác nhau cho các biểu mẫu đăng nhập có thể mất lượng thời gian khác nhau để xử lý. Việc xác định thời gian đầu vào khác nhau diễn ra trong các lần thử lặp lại bắt đầu tiết lộ các mẫu mà kẻ tấn công có thể ngoại suy và xây dựng từ đó \cite{wright-2023}.\\
\indent Có một số phần tốn thời gian khác nhau trong quá trình đăng nhập. Dài nhất thường là băm mật khẩu trước khi so sánh với hàm băm được lưu trữ hiện có. Các phần khác của quy trình cũng có thể mất một lượng thời gian có thể đo lường được, chẳng hạn như xác thực tên người dùng, xác định tên người dùng đó phù hợp với người dùng hợp lệ, v.v. Băm mật khẩu nói riêng là một tác vụ có chủ đích chậm, vì vậy khá dễ dàng để phân biệt sự khác biệt giữa một lần đăng nhập yêu cầu băm mật khẩu cho phù hợp, so với mật khẩu bỏ qua mật khẩu đó \cite{wright-2023}.\\
\indent Khi một lần đăng nhập thất bại xảy ra, nó có thể xảy ra nhanh hơn nhiều so với một lần đăng nhập thành công do người dùng không có tên đã cho hoặc hàm băm mật khẩu bị bỏ qua hoặc không khớp. Chỉ cần đo thời gian thực hiện với các lần đăng nhập khác nhau, thông tin tiềm ẩn về các lần đăng nhập có thể bị rò rỉ \cite{wright-2023}.\\
\underline{Các ví dụ về cách thức tấn công Timing Attack:}
\begin{itemize}
    \item Đoán mật khẩu : Kẻ tấn công có thể sử dụng Timing Attack để đoán mật khẩu bằng cáhc đo thời gian thực hiện của chức năng xác minh mật khẩu. Kẻ tấn công gửi 1 loạt các lần đoán mật khẩu đến hàm này và đo thời gian cần thực hiện của các lần đoán khác nhau, kẻ tấn công có thể xác định lần đoán nào gần nhất với mật khẩu đúng và sử dụng lần đoán đó làm lần đoán tiếp theo. Quá trình này có thể được lặp lại cho đến khi đoán được mật khẩu chính xác \cite{cqr-2023}.
    \item Khôi phục khóa mật mã: Bằng cách đo thời gian cần thiết để thực hiện các hoạt động mã hóa hoặc giải mã. Bằng cáhc gửi một loạt thông tin đầu vào đến hệ thống và đo thời gian thực hiện các thao tác, kẻ tấn công có thể suy ra thông tin về khóa bí mật và có khả năng khôi phục nó \cite{cqr-2023}.
    \item Bỏ qua xác thực: Kẻ tấn công có thể sử dụng một cuộc tấn công Timing Attack để vượt qua hệ thống xác thực bằng cách đo thời gian cần thiết để nhận được phản hồi cho yêu cầu đăng nhập. Bằng cách gửi một loạt các lần thử đăng nhập với tên người dung và mật khẩu khác nhau, kẻ tấn công có thể đo thời gian thực thi của hệ thống xác thực và xác định tổ hợp tên người dùng và mật khẩu nào gần đúng nhất. Quá trình này có thể được lặp lại cho đến khi đoán được thông tin thực chính xác, cho phép kẻ tấn công bỏ qua hệ thống xác thực \cite{cqr-2023}.

\end{itemize}
\subsubsection{Biện pháp tấn công và ngăn chặn}
\begin{itemize}
    \item Thêm nhiễu: Thêm các khoảng thời gian ngẫu nhiên vào quá trình xử lý để làm khó việc đoán thời gian thực thi chính xác.
    \item Sử dụng thuật toán hằng thời gian: Các thuật toán mật mã được thiết kế để thực thi trong một khoảng thời gian cố định, không phụ thuộc vào giá trị của dữ liệu hoặc khóa \cite{biswas2017survey}.
    \item Kiểm tra toàn bộ dữ liệu: Đảm bảo rằng toàn bộ dữ liệu đầu vào được kiểm tra trước khi phản hồi, thay vì dừng lại ngay khi phát hiện lỗi \cite{dhem2000practical}.
    \item Triển khai các hệ thống giới hạn tốc độ có thể ngăn kẻ tấn công thu thập đủ dữ liệu để thực hiện timing attack \cite{biswas2017survey}.
    \item  Nên xác định các chức năng quan trọng về bảo mật và tạo chúng theo thời gian không đổi \cite{dhem2000practical}.
\end{itemize}
