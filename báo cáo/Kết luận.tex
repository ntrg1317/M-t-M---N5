\subsection{Tổng kết Những việc đã làm được}
Trong bài báo cáo này, chúng em đã cung cấp một cái nhìn tổng quan về hệ mật khóa đối xứng và các phương pháp tấn công liên quan. Chúng em đã đi sâu vào các nguyên lý cơ bản của hệ mật khóa đối xứng, giải thích cách thức hoạt động và tầm quan trọng của nó trong lĩnh vực bảo mật thông tin. Đặc biệt, nhóm đã tập trung vào hai hệ mã hóa phổ biến nhất: DES (Data Encryption Standard) và AES (Advanced Encryption Standard), cung cấp cái nhìn tổng quan, cách thức hoạt động trong cả quá trình mã hóa và giải mã. Từ đó, nhóm đã làm rõ các điểm mạnh và yếu của từng phương pháp.\\

Nhóm đã giỡi thiệu tổng quan các trường hợp nổi bật của tấn công mã khóa đối xứng, bao gồm các kỹ thuật như Brute Force, tấn công Dictionary, tấn công trung gian (MITM), tấn công đã biết bản rõ (KPA), và tấn công chọn bản rõ (CPA), Tấn công Phân tích thời gian (Timing Attack). Mỗi kỹ thuật tấn công đều được phân tích cụ thể về cách thức hoạt động và mức độ hiệu quả khi đối phó với các hệ mã khóa khác nhau. Ngoài ra, nhóm còn tìm hiểu sâu hơn về hai kỹ thuật tấn công hệ mật DES là: Tấn công Vi phân (Differential Cryptanalysis), Tấn công Tuyến tính (Linear Cryptanalysis).\\

Trong phần xây dựng chương trình Demo, nhóm chúng em đã triển khai các chương trình minh họa quá trình mã hóa và giải mã sử dụng cả DES và AES. Để thực hiện điều này, chúng em đã phát triển các đoạn mã cụ thể mô phỏng từng bước của quá trình mã hóa và giải mã, từ việc nhập dữ liệu ban đầu, tạo khóa mã hóa, đến việc áp dụng các thuật toán mã hóa và cuối cùng là giải mã để khôi phục dữ liệu gốc. Các ví dụ này không chỉ giúp làm rõ cơ chế hoạt động của DES và AES mà còn minh chứng rõ ràng sự khác biệt về hiệu suất và độ bảo mật giữa hai phương pháp này.\\

Qua các chương trình Demo, chúng em đã thấy rằng trong khi AES thể hiện sự mạnh mẽ và phức tạp trong việc bảo vệ dữ liệu, DES lại bộc lộ nhiều điểm yếu, đặc biệt là với các tấn công Brute Force. Để minh họa điều này, em đã tiến hành một thử nghiệm cụ thể tấn công Brute Force đối với mã hóa DES. Trong thử nghiệm này, chúng em đã viết mã để thử tất cả các khóa có thể có trong không gian khóa của DES cho đến khi tìm ra khóa chính xác. Kết quả thử nghiệm cho thấy DES dễ bị tổn thương trước các tấn công kiểu này do kích thước khóa tương đối nhỏ (56 bit), khiến việc dò tìm khóa trở nên khả thi với khả năng tính toán hiện đại.\\

Thử nghiệm này không chỉ chứng minh tính dễ bị tổn thương của DES mà còn nhấn mạnh tầm quan trọng của việc sử dụng các thuật toán mã hóa có không gian khóa lớn hơn và cấu trúc phức tạp hơn như AES. AES, với kích thước khóa có thể là 128, 192 hoặc 256 bit, cung cấp một mức độ bảo mật cao hơn nhiều và rất khó bị phá vỡ bằng các tấn công Brute Force hiện tại.\\

Bài báo cáo này không chỉ cung cấp kiến thức cơ bản và chi tiết về các hệ mã khóa đối xứng và các kỹ thuật tấn công, mà còn minh chứng qua các ví dụ thực tế để làm rõ tầm quan trọng của việc chọn lựa các phương pháp mã hóa mạnh mẽ và an toàn hơn. Qua đó, em hy vọng đã mang lại một cái nhìn toàn diện và sâu sắc về bảo mật thông tin sử dụng hệ mật khóa đối xứng, đồng thời nhấn mạnh sự cần thiết của việc nghiên cứu và phát triển các phương pháp bảo mật tiên tiến để đối phó với các mối đe dọa ngày càng tinh vi.

\subsection{Hướng Nghiên cứu và Phát triển trong tương lai}
Lĩnh vực mật mã đối xứng đang không ngừng phát triển để đáp ứng nhu cầu bảo mật ngày càng cao trong kỷ nguyên số. Một số xu hướng nghiên cứu chính trong tương lai bao gồm:
\begin{itemize}
    \item \textbf{Máy tính lượng tử và Mật mã hậu lượng tử:} Sự phát triển của máy tính lượng tử đặt ra mối đe dọa nghiêm trọng đối với các thuật toán mật mã hiện tại, bao gồm cả mật mã đối xứng. Máy tính lượng tử có khả năng giải quyết các bài toán phức tạp một cách nhanh chóng, vượt trội so với các máy tính cổ điển, điều này làm suy yếu nhiều thuật toán mã hóa hiện có. Do đó, nghiên cứu đang tập trung vào việc phát triển các thuật toán mới có khả năng chống lại máy tính lượng tử, được gọi là "mật mã hậu lượng tử". Một số ví dụ hứa hẹn bao gồm mã hóa dựa trên mạng lưới, mã hóa dựa trên giao thức, và mã hóa dựa trên ký hiệu. Các thuật toán này được thiết kế để đảm bảo rằng ngay cả khi máy tính lượng tử trở nên phổ biến, dữ liệu được mã hóa vẫn sẽ được bảo vệ \cite{saberikamarposhti2024comprehensive}.
    \item \textbf{Tích hợp với học máy (Machine Learning):}
    \begin{itemize}
        \item \textbf{Đào tạo mô hình an toàn:} Đảm bảo rằng việc đào tạo các mô hình học máy trên hình ảnh mã hóa không làm ảnh hưởng đến tính riêng tư hay bảo mật. Điều này đòi hỏi các phương pháp đào tạo mới phải được phát triển, để các mô hình học máy có thể học từ dữ liệu mã hóa mà không cần giải mã nó, đảm bảo tính riêng tư của dữ liệu gốc.
        \item \textbf{Chống lại tấn công đối kháng:} Nghiên cứu ảnh hưởng của mã hóa đến khả năng chống lại các tấn công đối kháng của các mô hình học máy. Các tấn công đối kháng là các kỹ thuật được sử dụng để đánh lừa các mô hình học máy bằng cách cung cấp dữ liệu đầu vào được chế tác đặc biệt. Nghiên cứu trong lĩnh vực này sẽ tập trung vào việc làm thế nào để bảo vệ các mô hình học máy khỏi các tấn công này khi dữ liệu được mã hóa.
    \end{itemize}
    \item \textbf{Kết hợp với các kỹ thuật bảo mật khác:} Mật mã đối xứng thường được kết hợp với các kỹ thuật bảo mật khác để tạo ra hệ thống bảo mật toàn diện. Một trong những cách tiếp cận phổ biến là sử dụng các phương pháp tiếp cận lai, kết hợp mã hóa đối xứng và bất đối xứng cùng với các kỹ thuật xác thực và kiểm soát truy cập. Nghiên cứu nhằm phát triển các phương pháp tích hợp hiệu quả hơn giữa mật mã đối xứng và các kỹ thuật bảo mật khác, đồng thời giải quyết các vấn đề liên quan đến khả năng tương thích và quản lý khóa. Việc tích hợp này giúp tăng cường độ bảo mật và cung cấp các giải pháp bảo mật linh hoạt hơn cho các hệ thống phức tạp \cite{saberikamarposhti2024comprehensive}.
    \item \textbf{Ứng dụng trong các lĩnh vực mới:} Mật mã đối xứng có tiềm năng ứng dụng rộng rãi trong nhiều lĩnh vực mới như Internet vạn vật (IoT), xe tự lái, và trí tuệ nhân tạo. Các thiết bị IoT thường có tài nguyên hạn chế, do đó cần các giải pháp mật mã nhẹ và hiệu quả. Xe tự lái và trí tuệ nhân tạo yêu cầu bảo mật cao để bảo vệ dữ liệu và đảm bảo an toàn. Nghiên cứu tập trung vào việc phát triển các giải pháp mật mã phù hợp cho các yêu cầu cụ thể của từng lĩnh vực, đồng thời giải quyết các thách thức về bảo mật và quyền riêng tư trong môi trường mới này \cite{saberikamarposhti2024comprehensive}.
    \item Ngoài những xu hướng chính này, nghiên cứu trong tương lai về mật mã đối xứng cũng sẽ tập trung vào các lĩnh vực như bảo mật đám mây, bảo mật di động, và bảo mật dữ liệu. Bảo mật đám mây yêu cầu các giải pháp mật mã hiệu quả để bảo vệ dữ liệu được lưu trữ và truyền tải trong môi trường đám mây. Bảo mật di động cần các giải pháp bảo mật mạnh mẽ cho các thiết bị di động và các ứng dụng di động. Bảo mật dữ liệu yêu cầu các phương pháp bảo vệ dữ liệu toàn diện, từ lúc dữ liệu được tạo ra cho đến khi nó bị xóa \cite{saberikamarposhti2024comprehensive}.
\end{itemize}

Việc tiếp tục nghiên cứu và phát triển trong lĩnh vực này là rất cần thiết để đảm bảo an toàn thông tin trong kỷ nguyên số. Với sự phát triển không ngừng của khoa học máy tính và sự gia tăng nhu cầu bảo mật thông tin, lĩnh vực mật mã đối xứng chắc chắn sẽ tiếp tục phát triển và đóng vai trò quan trọng trong việc bảo vệ dữ liệu và hệ thống trong tương lai. Những tiến bộ trong nghiên cứu và ứng dụng sẽ giúp xây dựng các hệ thống bảo mật mạnh mẽ hơn, đảm bảo rằng thông tin của chúng ta được bảo vệ trong mọi hoàn cảnh.